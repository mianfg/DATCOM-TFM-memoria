% !TEX program = pdflatex
% !TEX encoding = UTF-8 Unicode

% Trabajo de Fin de Máster
% Máster en Ciencia de Datos e Ingeniería de Computadores
% Miguel Ángel Fernández Gutiérrez

\documentclass{scrbook}

\KOMAoptions{%
  fontsize=10pt,        % Tamaño de fuente
  paper=a4,             % Tamaño del papel
  headings=normal,      % Tamaño de letra para los títulos: small, normal, big
  parskip=half,         % Espacio entre párrafos: full (una línea) o half (media línea)
  headsepline=false,    % Una linea separa la cabecera del texto
  cleardoublepage=empty,% No imprime cabecera ni pie en páginas en blanco 
  chapterprefix=false,  % No antepone el texto "capítulo" antes del número
  appendixprefix=false, % No antepone el texto "Apéndice" antes de la letra
  listof=totoc,         % Añade a la tabla de contenidos la lista de tablas y figuras
  index=totoc,          % Añade a la talba de contenidos una entrada para el índice
  bibliography=totoc,	  % Añade a la tabla de contenidos una entrada para bibliografía
  %BCOR=5mm,            % Reserva margen interior para la encuadernación. 
                        % El valor dependerá el tipo de encuadernado y del grosor del libro.
  DIV=10,               % Cálcula el diseño de página según ciertos 
                        % parámetros. Al aumentar el número aumentamos el ancho de texto y disminuimos el ancho del margen. Una opción de 14 producirá márgenes estrechos y texto ancho.
  twoside=false
}

% INFORMACIÓN PARA LA VERSIÓN IMPRESA
% Si el documento ha de ser impreso en papel de tamaño a4 pero el tamaño del documento (elegido en \KOMAoptions con la ocpión paper) no es a4 descomentar la línea que carga el paquete `crop` más abajo. El paquete crop se encargará de centrar el documento en un a4 e imprimir unas guías de corte. El procedimiento completo para imprenta sería el siguiente:
% 0. Determinar, según el tipo de encuadernación del documento, el ancho reservado para el proceso de encuadernación (preguntar en la imprenta), es decir, la anchura del área del papel que se pierde durante el proceso de encuadernación. Fijar la varibale BCOR de \KOMAoptions a dicho valor.
% 1. Descomentar la siguiente línea e imprimir una única página con las guías de corte
% 2. Cambiar la opción `cross` por `cam` (o `off`) en el paquete crop y volver a compilar. Imprimir el documento (las guías de corte impresas no inferfieren con el texto).
% 3. Usar la página con las guías impresas en el punto 1 para cortar todas las páginas.

% \usepackage[a4, odd, center, pdflatex, cross]{crop} % Permite imprimir el documento en un a4 (si el tamaño es más pequeño) mostrando unas guías de corte. Útil para imprenta.

% VERSIÓN ELECTRÓNICA PARA TABLETA
% Las opciones siguientes seleccionan un tamaño de impresión similar a una tableta de 9 pulgadas con márgenes estrechos. Útil para producir una versión en pdf para ser leída en una tableta en lugar de impresa.
% Para que la portada quede centrada correctamente hay que editar el
% archivo `portada.tex` y eliminar el entorno `addmargin`

% \KOMAoptions{fontsize=10pt, paper=19.7104cm:14.7828cm, twoside=false, BCOR=0cm, DIV=14}

% ---------------------------------------------------------------------
%	PAQUETES 
% ---------------------------------------------------------------------

% CODIFICACIÓN E IDIOMA
% ---------------------------------------------------------------------
\usepackage[utf8]{inputenc} 			    % Codificación de caracteres

% Selección del idioma: cargamos por defecto inglés y español (aunque este último es el idioma por defecto para el documento). Cuando queramos cambiar de idioma escribiremos:
% \selectlanguage{english} o \selectlanguage{spanish}

\usepackage[english, spanish, es-nodecimaldot, es-noindentfirst, es-tabla]{babel}
\selectlanguage{spanish}

% Opciones cargadas para el paquete babel:
  % es-nodecimaldot: No cambia el punto decimal por una coma en modo matemático.
  % es-noindentfirst: No sangra los párrafos tras los títulos.
  % es-tabla: cambia el título del entorno `table` de "Cuadro" a "Tabla"

% Otras opciones del paquete spanish-babel:
  \unaccentedoperators % Desactiva los acentos en los operadores matemáticso (p.e. \lim, \max, ...). Eliminar esta opción si queremos que vayan acentuados

% MATEMÁTICAS
% ---------------------------------------------------------------------
\usepackage{amsmath, amsthm, amssymb, etoolbox, thmtools} % Paquetes matemáticas
\usepackage{mathtools}                % Añade mejoras a amsmath
\mathtoolsset{showonlyrefs=true}      % sólo se numeran las ecuaciones que se usan
\usepackage[mathscr]{eucal} 					% Proporciona el comando \mathscr para
                                      % fuentes de tipo manuscrito en modo matemático sin sobreescribir el comando \mathcal

% TIPOGRAFÍA 
% ---------------------------------------------------------------------
% El paquete microtype mejora la tipografía del documento.
\usepackage[activate={true,nocompatibility},final,tracking=true,kerning=true,spacing=true,factor=1100,stretch=10,shrink=10]{microtype}

% Las tipografías elegidas para el documento, siguiendo la guía de estilo de la UGR,
% son las siguientes
% Normal font: 			URW Palladio typeface. 
% Sans-serif font: 	Gill Sans
% Monospace font: 	Inconsolata
\usepackage[LGR, T1]{fontenc}
\usepackage[sc, osf]{mathpazo} \linespread{1.05}         
\usepackage[scaled=.95,type1]{cabin} % sans serif in style of Gill Sans
% Si el paquete cabin da error usar el siguiente comando en su lugar
% \renewcommand{\sfdefault}{iwona} 
\usepackage{inconsolata}
\usepackage{enumitem}
\setitemize{itemsep=4pt,topsep=4pt,parsep=4pt,partopsep=0pt}
\usepackage{dirtree}
\usepackage{caption}

% Selecciona el tipo de fuente para los títulos (capítulo, sección, subsección) del documento.
\setkomafont{disposition}{\sffamily\bfseries}

% Cambia el ancho de la cita. Al inicio de un capítulo podemos usar el comando \dictum[autor]{cita} para añadir una cita famosa de un autor.
\renewcommand{\dictumwidth}{0.45\textwidth} 

\recalctypearea % Necesario tras definir la tipografía a usar.

\usepackage{setspace}
\usepackage{changepage}   % for the adjustwidth environment

% TABLAS, GRÁFICOS Y LISTADOS DE CÓDIGO
% ---------------------------------------------------------------------
\usepackage{booktabs}
% \renewcommand{\arraystretch}{1.5} % Aumenta el espacio vertical entre las filas de un entorno tabular

\usepackage{tikz}
\usepackage{tikz-cd}
\usepackage{float}
\usepackage{rotating}
\usepackage{svg}
\usepackage{xcolor, graphicx}
\usepackage{pdflscape}
\usepackage{stfloats,framed}
\usepackage{booktabs}
\usepackage{mathabx}

\usepackage{tabularx}

\newcolumntype{L}[1]{>{\hsize=#1\hsize\RaggedRight} X}


% Carpeta donde buscar los archivos de imagen por defecto
\graphicspath{{img/}}

% IMAGEN DE LA PORTADA
% Existen varias opciones para la imagen de fondo de la portada del TFG. Todas ellas tienen en logotipo de la universidad de Granada en la cabecera. Las opciones son las siguientes:
% 1. portada-ugr y portada-ugr-color: diseño con marca de agua basada en el logo de la UGR (en escala de grises y color).
% 2. portada-ugr-sencilla y portada-ugr-sencilla-color: portada únicamente con el logotipo de la UGR en la cabecera.
\usepackage{eso-pic}
\newcommand\BackgroundPic{%
	\put(0,-1){%
		\parbox[b][\paperheight]{\paperwidth}{%
			\vfill
			\centering
      % Indicar la imagen de fondo en el siguiente comando
			\includegraphics[width=\paperwidth,height=\paperheight,%
			keepaspectratio]{portada-ugr-color-no-margin-bottom}%
			\vfill
}}}

\usepackage{listings} % Para la inclusión de trozos de código
\lstset{
     literate=%
         {á}{{\'a}}1
         {í}{{\'i}}1
         {é}{{\'e}}1
         {ý}{{\'y}}1
         {ú}{{\'u}}1
         {ó}{{\'o}}1
         {ě}{{\v{e}}}1
         {š}{{\v{s}}}1
         {č}{{\v{c}}}1
         {ř}{{\v{r}}}1
         {ž}{{\v{z}}}1
         {ď}{{\v{d}}}1
         {ť}{{\v{t}}}1
         {ň}{{\v{n}}}1                
         {ů}{{\r{u}}}1
         {Á}{{\'A}}1
         {Í}{{\'I}}1
         {É}{{\'E}}1
         {Ý}{{\'Y}}1
         {Ú}{{\'U}}1
         {Ó}{{\'O}}1
         {Ě}{{\v{E}}}1
         {Š}{{\v{S}}}1
         {Č}{{\v{C}}}1
         {Ř}{{\v{R}}}1
         {Ž}{{\v{Z}}}1
         {Ď}{{\v{D}}}1
         {Ť}{{\v{T}}}1
         {Ň}{{\v{N}}}1                
         {Ů}{{\r{U}}}1    
}
\usepackage{lstautogobble}  % Fix relative indenting
\usepackage{zi4}            % Nice font

\definecolor{bluekeywords}{rgb}{0.13, 0.13, 1}
\definecolor{greencomments}{rgb}{0, 0.5, 0}
\definecolor{redstrings}{rgb}{0.9, 0, 0}
\definecolor{graynumbers}{rgb}{0.5, 0.5, 0.5}

\lstset{
    autogobble,
    columns=fullflexible,
    showspaces=false,
    showtabs=false,
    breaklines=true,
    showstringspaces=false,
    breakatwhitespace=true,
    escapeinside={(*@}{@*)},
    commentstyle=\color{greencomments},
    keywordstyle=\color{bluekeywords},
    stringstyle=\color{redstrings},
    numberstyle=\color{graynumbers},
    basicstyle=\ttfamily\footnotesize,
    frame=l,
    framesep=30pt,
    xleftmargin=30pt,
    tabsize=4,
    captionpos=b,
    basicstyle=\ttfamily,
    numbers=left,
    numberstyle=\footnotesize\color{graynumbers},
    numbersep=10pt,
    escapechar=|,
    numberblanklines=false,
    columns=flexible
}

\renewcommand{\lstlistingname}{Programa}% Listing -> Algorithm
\renewcommand{\lstlistlistingname}{List of \lstlistingname s}% List of Listings -> List of Algorithms

\usepackage{newfloat}
\DeclareFloatingEnvironment[listname={Índice de problemas},placement=H]{problema}
\DeclareFloatingEnvironment[listname={Índice de tablas},placement=H]{tabla}

% CABECERAS
% ---------------------------------------------------------------------
% Si queremos modificar las cabeceras del documento podemos usar el paquete
% `scrlayer-scrpage` de KOMA-Script. Consultar la documentación al respecto.
% \usepackage[automark]{scrlayer-scrpage}

% VARIOS
% ---------------------------------------------------------------------

%\usepackage{showkeys}	% Muestra las etiquetas del documento. Útil para revisar las referencias cruzadas.

\usepackage[style=apa,backend=biber,sorting=nty,citestyle=numeric]{biblatex}
\usepackage{hyperref}
\hypersetup{%
  % hidelinks,            % Enlaces sin color ni borde. El borde no se imprime
  linkbordercolor=0.8 0 0,
  citebordercolor=0 0.8 0,
  citebordercolor=0 0.8 0,
  colorlinks = true,            % Color en texto de los enlaces. Comentar esta línea o cambiar `true` por `false` para imprimir el documento.
  linkcolor = [rgb]{0.5, 0, 0}, % Color de los enlaces internos
  urlcolor = [rgb]{0, 0, 0.5},  % Color de los hipervínculos
  citecolor = [rgb]{0, 0.5, 0}, % Color de las referencias bibliográficas
	pdftitle={\varDocumento},%
	pdfauthor={\textcopyright\ \miNombre, \miFacultad, \miUniversidad},%
  pdfsubject={\varDocumento},%
	pdfkeywords={},%
	pdfcreator={pdfLaTeX},%
}
\addbibresource{references.bib}

\DeclareFieldFormat{labelnumberwidth}{\mkbibbrackets{#1}}

\defbibenvironment{bibliography}
  {\list
     {\printtext[labelnumberwidth]{%
      \printfield{labelprefix}%
      \printfield{labelnumber}}}
     {\setlength{\labelwidth}{\labelnumberwidth}%
      \setlength{\leftmargin}{\labelwidth}%
      \setlength{\labelsep}{\biblabelsep}%
      \addtolength{\leftmargin}{\labelsep}%
      \setlength{\itemsep}{8pt}%
      \setlength{\parsep}{\bibparsep}}%
      \renewcommand*{\makelabel}[1]{\hss##1}}
  {\endlist}
  {\item}

% ---------------------------------------------------------------------
% COMANDOS Y ENTORNOS
% ---------------------------------------------------------------------

% DEFINICIÓN DE COMANDOS Y ENTORNOS

% CONJUNTOS DE NÚMEROS

\newcommand{\N}{\mathbb{N}}     % Naturales
\newcommand{\R}{\mathbb{R}}     % Reales
\newcommand{\Z}{\mathbb{Z}}     % Enteros
\newcommand{\Q}{\mathbb{Q}}     % Racionales
\newcommand{\C}{\mathbb{C}}     % Complejos

% TEOREMAS Y ENTORNOS

\newtheorem*{teorema*}{Teorema}
\newtheorem{teorema}{Teorema}[chapter]
\newtheorem{proposicion}{Proposición}[chapter]
\newtheorem{lema}{Lema}[chapter]
\newtheorem{corolario}{Corolario}[chapter]

\theoremstyle{definition}
\newtheorem{definicion}{Definición}[chapter]
\newtheorem{ejemplo}{Ejemplo}[chapter]
\newtheorem{sistemaformal}{Sistema formal}[chapter]
\newtheorem{sistemalogico}[sistemaformal]{Sistema lógico}

\theoremstyle{remark}
\newtheorem{observacion}{Observación}[chapter]
\newtheorem{nota}{Nota}[definicion]

\newcommand\Vtextvisiblespace[1][.3em]{%
\mbox{\kern.06em\vrule height.3ex}%
\vbox{\hrule width#1}%
\hbox{\vrule height.3ex}}

\newcommand{\cabezal}[1]{\overset{\blacktriangledown}{#1}}
\newcommand{\palabra}[1]{\texttt{\textquotesingle{}{#1}\textquotesingle{}}}

\makeatletter
\newenvironment{proofw}{\par
\pushQED{\qed}%
\normalfont \topsep6\p@\@plus6\p@\relax
\trivlist
\item[]\ignorespaces
}{%
\popQED\endtrivlist\@endpefalse
}
\makeatother

\makeatletter
\def\ll@definicion{%
\protect\numberline{\csname the\thmt@envname\endcsname}%
\ifx\@empty\thmt@shortoptarg
  \thmt@thmname
\else
  \thmt@shortoptarg
\fi}
\def\l@thmt@definicion{} 
\makeatother

\makeatletter
\def\ll@sistemaformal{%
\protect\numberline{\csname the\thmt@envname\endcsname}%
\ifx\@empty\thmt@shortoptarg
  \thmt@thmname
\else
  \thmt@shortoptarg
\fi}
\def\l@thmt@sistemaformal{} 
\makeatother

\makeatletter
\def\ll@sistemalogico{%
\protect\numberline{\csname the\thmt@envname\endcsname}%
\ifx\@empty\thmt@shortoptarg
  \thmt@thmname
\else
  \thmt@shortoptarg
\fi}
\def\l@thmt@sistemalogico{} 
\makeatother

% --------------------------------------------------------------------
% INFORMACIÓN DEL TFM Y EL AUTOR
% --------------------------------------------------------------------
\usepackage{xspace} % Para problemas de espaciado al definir comandos

\newcommand{\miBorrador}{BORRADOR}
\newcommand{\varDocumento}{Trabajo de Fin de Máster\xspace}
\newcommand{\varTitulo}{Título\xspace}
\newcommand{\varSubtitulo}{Traducción título\xspace}
\newcommand{\miNombre}{Miguel Ángel Fernández Gutiérrez\xspace}
\newcommand{\miGrado}{Máster Universitario en Ciencia de Datos \\ e Ingeniería de Computadores}
\newcommand{\miFacultad}{Escuela Técnica Superior de \\ Ingenierías Informática y de Telecomunicación}
\newcommand{\miFacultadNoSpace}{Escuela Técnica Superior de Ingenierías Informática y de Telecomunicación}
\newcommand{\miUniversidad}{Universidad de Granada}
% Añadir tantos tutores como sea necesario separando cada uno de ellos
% mediante el comando `\medskip` y una línea en blanco
\newcommand{\miTutor}{
  Isaac Triguero Velázquez \\ Daniel Molina Cabrera \\ \emph{Departamento de Ciencias de la Computación e Inteligencia Artificial}
}
\newcommand{\miCurso}{2024/2025\xspace}

% HYPERREFERENCES
% --------------------------------------------------------------------
\usepackage{xurl}
\usepackage[noabbrev,capitalize,nameinlink]{cleveref}

\crefname{section}{sección}{secciones}
\Crefname{section}{sec.}{secs.}
\crefname{chapter}{capítulo}{capítulos}
\Crefname{chapter}{cap.}{caps.}
\crefname{part}{parte}{partes}
\Crefname{part}{parte}{partes}

\crefname{programa}{programa}{programas}
\Crefname{programa}{Programa}{Programas}
\crefname{problema}{problema}{problemas}
\Crefname{problema}{Problema}{Problemas}
\crefname{ejemplo}{ejemplo}{ejemplos}
\Crefname{ejemplo}{Ejemplo}{Ejemplos}
\crefname{proposicion}{proposición}{proposiciones}
\Crefname{proposicion}{prop.}{props.}
\crefname{lema}{lema}{lemas}
\Crefname{lema}{Lema}{Lemas}
\crefname{corolario}{corolario}{corolarios}
\Crefname{corolario}{Corolario}{Corolarios}
\crefname{teorema}{teorema}{teoremas}
\Crefname{teorema}{Teorema}{Teoremas}
\crefname{programa}{programa}{programas}
\Crefname{programa}{Programa}{Programas}
\crefname{definicion}{definición}{definiciones}
\Crefname{definicion}{Definición}{Definiciones}
\crefname{figure}{figura}{figuras}
\Crefname{figure}{Figura}{Figuras}
\crefname{tabla}{tabla}{tablas}
\Crefname{tabla}{Tabla}{Tablas}
\crefname{appendix}{apéndice}{apéndices}
\Crefname{appendix}{Apéndice}{Apéndices}
\crefname{listing}{programa}{programas}
\Crefname{listing}{Programa}{Programas}
\crefname{line}{línea}{líneas}
\Crefname{line}{Línea}{Líneas}
\crefname{page}{página}{páginas}
\Crefname{page}{pág.}{págs.}
\crefname{sistemalogico}{sistema lógico}{sistemas lógicos}
\Crefname{sistemalogico}{Sistema Lógico}{Sistemas Lógicos}
\crefname{sistemaformal}{sistema formal}{sistemas formales}
\Crefname{sistemaformal}{Sistema Formal}{Sistemas Formales}
\newcommand{\crefpairconjunction}{ y }
\newcommand{\crefrangeconjunction}{ a }


%index
\usepackage{makeidx}
\makeindex

\hypersetup{
 pdfauthor={Fernández Gutiérrez, Miguel Ángel},
% pdftitle={[\miBorrador] — \miTitulo},
 pdftitle={\varDocumento},
 pdfsubject={\varDocumento — \miGrado — Curso \miCurso},
 pdfkeywords={TFM, ciencia de datos},
 pdfproducer=mianfg,
 pdfcreator=pdflatex,
 breaklinks=true
}

\begin{document}

% --------------------------------------------------------------------
% FRONTMATTER
% -------------------------------------------------------------------
\frontmatter % Desactiva la numeración de capítulos y usa numeración romana para las páginas
%\flushbottom
% \pagestyle{plain} % No imprime cabeceras

% !TeX root = ../libro.tex
% !TeX encoding = utf8

%*******************************************************
% Titlepage
%*******************************************************
\begin{titlepage}
  \AddToShipoutPicture*{\BackgroundPic}
  \phantomsection 
  \pdfbookmark[1]{Portada}{title}

  % Para que el título esté centrado en la página.
  % Los valores numéricos deberán elegirse de acuerdo con el diseño de
  % página (sobre todo si se cambia la opción BCOR o DIV).
  % antes 2.575cm
  \begin{addmargin}[2cm]{0cm}
  \begin{flushleft}
    \Large  
    \hfill\vfil

    \large{\textsf{\miFacultad}}
    \vfill

    {\large\textsc\miGrado} \vfill


    {\large\textsc{Trabajo de Fin de Máster}}

    \begin{flushleft}
      \Huge
      \setstretch{0.8}
      \varTitulo\\[18pt]
      \huge
      \varSubtitulo
    \end{flushleft}

    \vfill\vfill\vfill\vfill

    \textsf{\normalsize{Presentado por:}}\\
    {\normalsize\textrm{\miNombre}} 
    \bigskip

    \textsf{\normalsize{Tutores:}}\\
    {\normalsize\rmfamily\miTutor}

    \bigskip
    \textsf{\normalsize{Curso académico \miCurso}}
    \vspace*{-2cm}
  \end{flushleft}  
  \end{addmargin}       

\end{titlepage}   
\cleardoublepage
\endinput

% !TeX root = ../libro.tex
% !TeX encoding = utf8

%*******************************************************
% Little Dirty Titlepage
%*******************************************************

\thispagestyle{empty}

\begin{center}
  \large  

  \vspace*{\stretch{1}}

  \begingroup
  \Large{\varDocumento} \\[18pt]
  \huge{\varTitulo}
  \bigskip
  \endgroup

  \textrm{\miNombre}

  \vspace{\stretch{5}}

\end{center}  

\newpage
\thispagestyle{empty}

\hfill

\vfill

\textbf{Alumno}\\
\miNombre
\medskip

\emph{\varTitulo\\(\varSubtitulo)}\\
Trabajo de Fin de Máster. Curso académico \miCurso
\bigskip

\begin{minipage}[t]{0.45\textwidth} %0.25
  \flushleft
  \textbf{Responsables de tutorización}\\
  \medskip
  \miTutor
\end{minipage}
\begin{minipage}[t]{0.10\textwidth} %0.25
  \hspace{1cm}
\end{minipage}
%\begin{minipage}[t]{0.45\textwidth}
%  \flushleft
%  \miTutor
%\end{minipage}
\begin{minipage}[t]{0.45\textwidth} %0.30
  \flushright
  \textbf{\miGrado}
  \medskip

  \miFacultadNoSpace
  \medskip

  \miUniversidad
\end{minipage}

\vspace*{2em}

This work is part of the project \textit{``Inteligencia Artificial Ética, Responsable y de Propósito General: Aplicaciones En Escenarios De Riesgo''  (IAFER)}. Exp.: TSI-$100927$-$2023$-$1$

\newpage
\endinput

%% !TeX root = ../libro.tex
% !TeX encoding = utf8
%
%*******************************************************
% Declaración de originalidad
%*******************************************************

\thispagestyle{empty}

\hfill\vfill
\vfill

%\textsc{Declaración de originalidad}\\\bigskip

%D. \miNombre \\\medskip

Declaro explícitamente que el trabajo presentado como Trabajo de Fin de Máster (TFM), correspondiente al curso académico \miCurso, es original, entendida esta, en el sentido de que no he utilizado para la elaboración del trabajo fuentes sin citarlas debidamente.
\medskip

En Granada, a \today \\[4pt]
\begin{flushleft} 
Fdo: \miNombre 

%\includegraphics[width=6cm,keepaspectratio]{firma.png}

\end{flushleft}

\vfill

\cleardoublepage
\endinput

%\input{preliminares/dedicatoria}                % Opcional
\input{preliminares/tablacontenidos}

%\input{preliminares/agradecimientos}            % Opcional

% \pagestyle{scrheadings} % A partir de ahora sí imprime cabeceras

%% !TeX root = ../main.tex
% !TeX encoding = utf8
%
%*******************************************************
% Summary
%*******************************************************

\selectlanguage{english}
\chapter{Summary}

The digitalization of the process industry has created a data-rich but knowledge-sparse environment, where terabytes of untagged sensor data are generated daily while the critical operational logic required to define safety anomalies remains trapped in unstructured technical documentation. This manual dependency creates a severe bottleneck, as interpreting these complex specifications requires domain engineers, making the process unscalable. While the literature is rich in data-driven anomaly detection methods that rely on historical patterns, there is a notable scarcity of frameworks that address the \textit{definition} of anomalies themselves. Traditional methods fail to identify failures that are described in safety manuals but have never occurred in operation, as they are absent from training data.

To address this definition gap, we propose a multi-stage semantic compilation framework that treats natural language specifications as source code, formalized into executable rules and deployed on a lightweight streaming engine. By leveraging a Program-of-Thoughts paradigm and layout-aware Retrieval-Augmented Generation, we transform ambiguous operating manuals into deterministic Python logic, effectively treating the Large Language Model not as an oracle, but as a compiler for diverse engineering constraints.

Experimental results on a synthetic dataset demonstrate that this approach achieves near-perfect validity in rule generation through constrained decoding, and significantly improves semantic recall for complex temporal constraints compared to standard baselines. Crucially, the architecture explicitly decouples the probabilistic definition layer from the execution layer. We rely on local models solely for the extraction process to guarantee data sovereignty, while the runtime execution remains purely deterministic.

Validation on a streaming runtime indicates that the extracted rules can be evaluated efficiently using incremental statistics. These results serve as a proof-of-concept for high-frequency industrial edge deployment, offering a foundation for more autonomous, specification-driven monitoring systems.

\vspace{2em}

\begin{description}
\item[Keywords:] Anomaly Definition, Large Language Models (LLMs), Retrieval-Augmented Generation (RAG), Search-Augmented Generation (SAG), Structured Output, Industrial IoT, Stream Processing, Program-of-Thoughts, Knowledge Extraction, Edge Computing.
\end{description}


\endinput

%% !TeX root = ../main.tex
% !TeX encoding = utf8
%
%*******************************************************
% Summary
%*******************************************************

\selectlanguage{spanish}
\chapter{Resumen}

La digitalización de la industria de procesos ha creado un entorno rico en datos pero pobre en conocimiento, donde diariamente se generan terabytes de datos de sensores sin etiquetar, mientras que la lógica operativa crítica necesaria para definir anomalías de seguridad permanece atrapada en documentación técnica no estructurada. Esta dependencia manual crea un grave cuello de botella, ya que la interpretación de estas especificaciones complejas requiere ingenieros de dominio, haciendo que el proceso sea poco escalable. Los métodos tradicionales de detección de anomalías basados en datos dependen de patrones históricos, lo que los hace insuficientes para identificar modos de fallo ``zero-shot'' en entornos diseñados para la seguridad (\textit{safe-by-design}), donde los eventos catastróficos son, por definición, raros y están ausentes de los datos de entrenamiento.

Para abordar este ``cuello de botella de definición'', proponemos un marco de compilación semántica de múltiples etapas que trata las especificaciones en lenguaje natural como código fuente, formalizándolas en reglas ejecutables y desplegándolas en un motor de streaming ligero. Aprovechando el paradigma ``Program-of-Thoughts'' y la Generación Aumentada por Recuperación (RAG) consciente del diseño, transformamos manuales operativos ambiguos en lógica Python determinista, tratando efectivamente al Modelo de Lenguaje Grande (LLM) no como un oráculo, sino como un compilador de diversas restricciones de ingeniería.

Los resultados experimentales en un conjunto de datos ``gemelo'' sintético demuestran que este enfoque logra una validez sintáctica del 100\% en la generación de reglas mediante decodificación restringida, y mejora significativamente la recuperación semántica (\textit{recall}) para restricciones temporales complejas en comparación con las líneas base \textit{zero-shot} estándar. Crucialmente, la arquitectura desacopla explícitamente la capa de definición probabilística de la capa de ejecución. Confiamos en modelos locales únicamente para el proceso de extracción para garantizar la soberanía de los datos, mientras que la ejecución en tiempo de ejecución permanece puramente determinista.

La validación en un entorno de ejecución de streaming demuestra que las reglas extraídas pueden ser evaluadas en tiempo $O(1)$ utilizando estadísticas incrementales, confirmando su idoneidad para el despliegue en entornos industriales de alta frecuencia en el borde (\textit{edge}).

\vspace{2em}

\begin{description}
\item[Palabras clave:] Definición de Anomalías, Modelos de Lenguaje Grande (LLM), Generación Aumentada por Recuperación (RAG), Generación Aumentada por Búsqueda (SAG), Salida Estructurada, IoT Industrial, Procesamiento de Flujos, Program-of-Thoughts, Extracción de Conocimiento, Computación en el Borde (Edge Computing).
\end{description}

\endinput


% --------------------------------------------------------------------
% MAINMATTER
% --------------------------------------------------------------------
\mainmatter % activa la numeración de capítulos, resetea la numeración de las páginas y usa números arábigos

\input{parts/part-01}

% --------------------------------------------------------------------
% APPENDIX: Opcional
% --------------------------------------------------------------------

%\appendix % Reinicia la numeración de los capítulos y usa letras para numerarlos
%\pdfbookmark[-1]{Apéndices}{appendix} % Alternativamente podemos agrupar los apéndices con un nuevo \part{Apéndices}
%\cleardoublepage\part*{Apéndices}
%\addcontentsline{toc}{part}{Apéndices}

%% !TeX root = ../libro.tex
% !TeX encoding = utf8

\chapter{Sobre el código de este trabajo}\label{ap:codigo-trabajo}

Junto a esta memoria, el trabajo incluye un repositorio de código accesible desde:
\begin{adjustwidth}{30pt}{}
    \href{https://github.com/mianfg-DGIIM/TFG}{https://github.com/mianfg-DGIIM/TFG}
\end{adjustwidth}
El repositorio tiene la siguiente estructura de archivos y directorios:
\vspace{8pt}
\dirtree{%
.1 /TFG.
.2 memoria\DTcomment{Memoria del trabajo}.
.3 memoria.pdf\DTcomment{Archivo PDF de esta memoria}.
.3 tex\DTcomment{Contiene el código en \LaTeX{ }{ }de la memoria}.
.4 memoria.tex.
.4 ....
.2 codigo\DTcomment{Todo el código del trabajo}.
.3 maquinas\_turing\DTcomment{Contiene descripciones de máquinas de Turing}.
.4 mas\_a\_que\_b.mt.
.4 ....
.3 adivina\_consistente.py.
.3 ....
}
\vspace{8pt}
En la carpeta \texttt{codigo} se encuentran:
\begin{itemize}
    \item Por una parte, las descripciones de máquinas de Turing comentadas en la \cref{prop:python-a-monocinta}, como archivos \texttt{.mt} dentro de la carpeta \texttt{maquinas\_turing}.
    \item Los programas desarrollados en este trabajo, junto con algunos otros.
\end{itemize}

\section{Cómo ejecutar los programas}

Los programas de la carpeta \texttt{codigo} pueden ejecutarse importando las funciones necesarias en el intérprete de Python. En la librería \texttt{utilidades.py} aparecen múltiples funciones que se usan en varios programas, y que son útiles para poder ejecutar otros. A modo de ejemplo, si queremos ejecutar la función \texttt{simula\_turing\_mult} de \texttt{simula\_turing.py}, haremos:
\begin{lstlisting}[numbers=none,frame=none]
TFG/codigo$ python
>>> from utilidades import leer
>>> from simula_turing import simula_turing_mult
>>> codificacion = leer('./maquinas_turing/mas_a_que_b.mt')
>>> entrada = 'abaaabbb'
>>> simula_turing_mult(codificacion, entrada)
'q_0 : X X X X X X X X [_] (rechaza)'
\end{lstlisting}

\section{Índice de programas}

%\footnotesize
\begin{tabularx}{\textwidth}{L{0.3} L{0.5} L{0.2}}
\midrule
Programa & Descripción & ¿Ejecutable? \\
\midrule
\texttt{adivina\_consistente.py} \linebreak \small{(\cref{lst:adivina-consistente})} & Usado en la prueba del \cref{teo:incompletitud-sintactica}: de ser el sistema lógico completo, este programa decidiría \hyperref[prob:adivina-consistente]{\textsc{AdivinaConsistente}} \vspace{4pt} & no [NE1] \\

\texttt{c\_d.py} \linebreak \small{(\cref{lst:c-d})} & Usado para probar que si un programa es decidible, su complementario también (\cref{prop:c-decidible}) \vspace{4pt} & no [NE2] \\

\texttt{c\_diagonal.py}$^\lozenge$ \linebreak \small{(\cref{lst:c-diagonal})} & Si \hyperref[prob:diagonal]{\textsc{Diagonal}} fuese decidible, este programa decidiría el problema \hyperref[prob:c-diagonal]{\textsc{C-Diagonal}} (\cref{prop:universal-no-decidible}) \vspace{4pt} & sí \\

\texttt{diagonal.py}$^\lozenge$ \linebreak \small{(\cref{lst:diagonal})} & Si \hyperref[prob:universal]{\textsc{Universal}} fuese decidible, este programa decidiría \hyperref[prob:diagonal]{\textsc{Diagonal}} (\cref{prop:universal-no-decidible}) \vspace{4pt} & sí \\

\texttt{diagonal\_a\_universal.py} \linebreak \small{(\cref{lst:diagonal-a-universal})} & Implementa la reducción de \hyperref[prob:diagonal]{\textsc{Diagonal}} a \hyperref[prob:universal]{\textsc{Universal}} \vspace{4pt} & sí \\

\texttt{es\_prueba\_peano.py} & Este programa es capaz de comprobar si una secuencia de fórmulas es prueba de otra fórmula en \hyperref[sl:peano]{\textbf{Peano}} \vspace{4pt} & no impl. [NI1] \\

\texttt{es\_teorema\_peano.py}$^\lozenge$ \linebreak \small{(\cref{lst:es-teorema-peano})} & Este programa hace \hyperref[prob:es-teorema-peano]{\textsc{EsTeoremaPeano}} semidecidible (\cref{prop:es-teorema-peano-semidecidible}) \vspace{4pt} & no [NE3] \\

\texttt{es\_teorema\_S.py} \linebreak \small{(\cref{lst:es-teorema-s})} & Prueba que \hyperref[prob:es-teorema]{\textsc{EsTeorema}} es decidible para sistemas lógicos sintácticamente consistentes (\cref{prop:es-teorema-consistente-decidible}) \vspace{4pt} & no [NE4] \\

\texttt{funcion\_con\_objeto}\linebreak\texttt{\_codificado.py} \linebreak \small{(\cref{lst:funcion-con-objeto-codificado})} & Este programa muestra que los programas SISO no imponen una restricción sobre los programas que podemos decidir: podemos aceptar como entrada y devolver como salida cualquier programa debidamente codificado (en este caso, mediante la librería nativa \texttt{picke}) \vspace{4pt} & sí \\

\texttt{godel.py} \linebreak \small{(\cref{lst:godel})} & Programa central en la demostración de la versión semántica del Primer Teorema de Incompletitud en el caso general (\cref{teo:incompletitud-semantica}) \vspace{4pt} & no [NE5] \\

\texttt{godel\_peano.py}$^\lozenge$ \linebreak \small{(\cref{lst:godel-peano})} & Programa central en la demostración de la versión semántica del Primer Teorema de Incompletitud para \hyperref[sl:peano]{\textbf{Peano}} (\cref{teo:incompletitud-peano}) \vspace{2pt} & no [NE6] \\

\midrule
\end{tabularx}
\vspace*{-0.2cm}
\begin{tabla}
\caption{Índice de programas usados en este trabajo}
\label{tab:indice-programas}
\end{tabla}
\vspace*{-1cm}

\newpage

\vspace*{0.9cm}

\begin{tabularx}{\textwidth}{L{0.3} L{0.5} L{0.2}}
\midrule
Programa & Descripción & ¿Ejecutable? \\
\midrule
\texttt{ignora\_entrada.py}$^\lozenge$ \linebreak \small{(\cref{lst:ignora-entrada})} \vspace{4pt} & Permite ignorar la entrada y ejecutar programas y entradas almacenados en disco & sí \\

\texttt{maquina\_universal.py} \linebreak \small{(\cref{lst:maquina-universal})} & Implementa una máquina universal en Python, es decir, una función que acepta como entradas un programa en Python y una entrada, y ejecuta el programa en Python con dicha entrada \vspace{4pt} & sí \\

\texttt{maquina\_universal}\linebreak\texttt{\_parada.py}$^\lozenge$ \linebreak \small{(\cref{lst:maquina-universal-parada})} \vspace{4pt} & Programa usado para demostrar que \hyperref[prob:parada]{\textsc{Parada}} es no decidible (\cref{prop:parada-no-decidible}) & sí \\

\texttt{mas\_a\_que\_b.py} \linebreak \small{(\cref{lst:mas-a-que-b})} & Este programa decide el problema \hyperref[prob:mas-a-que-b]{\textsc{MásAQueB}} (\cref{prop:masaqueb-decidible}) \vspace{4pt} & sí \\

\texttt{mas\_a\_que\_b\_v2.py} \linebreak \small{(\cref{lst:mas-a-que-b-v2})} & Este programa decide el problema \hyperref[prob:mas-a-que-b]{\textsc{MásAQueB}}, aceptando cualquier palabra como entrada \vspace{4pt} & sí \\

\texttt{modifica\_adivina}\linebreak\texttt{\_consistente.py}  \linebreak \small{(\cref{lst:modifica-adivina-consistente})} & Usado para probar la no decidibilidad de \hyperref[prob:adivina-consistente]{\textsc{AdivinaConsistente}} (\cref{prop:adivina-consistente-no-decidible}) \vspace{4pt} & no [NE7] \\

\texttt{parada\_a\_parada\_en}\linebreak\texttt{\_vacio.py} \linebreak \small{(\cref{lst:parada-a-parada-en-vacio})} \vspace{4pt} & Implementa la reducción de \hyperref[prob:parada]{\textsc{Parada}} a \hyperref[prob:parada-en-vacio]{\textsc{ParadaEnVacío}} & no [NE8] \\

\texttt{parada\_a\_peano.py} & Traduce una afirmación del tipo \emph{``$P$ para con entrada vacía''} a una fórmula de \hyperref[sl:peano]{\textbf{Peano}} \vspace{4pt} & no impl. [NI2] \\

\texttt{parada\_en\_vacio\_a\_es}\linebreak\texttt{\_verdadero\_peano.py}$^\lozenge$ \linebreak \small{(\cref{lst:parada-en-vacio-a-es-verdadero-peano})} \vspace{4pt} & Implementa la reducción de \hyperref[prob:parada-en-vacio]{\textsc{ParadaEnVacío}} a \hyperref[prob:es-verdadero-peano]{\textsc{EsVerdaderoPeano}} & no [NE9] \\

\texttt{si.py} \linebreak \small{(\cref{lst:si})} \vspace{4pt} & Este programa decide el problema \hyperref[prob:si]{\textsc{Sí}} & sí \\

\texttt{simula\_turing.py} \linebreak \small{(\cref{lst:simula-turing})} & Este programa simula una máquina de Turing. Puede usarse con cualquiera de las máquinas de la carpeta \texttt{maquinas\_turing} \vspace{4pt} & sí \\

\texttt{simula\_turing}\linebreak\texttt{\_decision.py} & Este programa es una variación de \texttt{simula\_turing.py} para hacer que el programa sea de decisión (siempre se devuelve \palabra{sí} o \palabra{no}). Este programa no se usa en el trabajo, pero se deja para experimentación \vspace{2pt} & sí \\
\midrule
\end{tabularx}
\vspace*{-0.2cm}
\begin{tabla}
\caption*{Tabla A.1. (cont.): Índice de programas usados en este trabajo}
\label{tab:indice-programas}
\end{tabla}


\newpage

\vspace*{0.9cm}

\begin{tabularx}{\textwidth}{L{0.3} L{0.5} L{0.2}}
\midrule
Programa & Descripción & ¿Ejecutable? \\
\midrule
\texttt{suma\_binaria.py}$^\lozenge$ & Este programa decide los problemas \hyperref[prob:es-teorema]{\textsc{EsTeorema}} y \hyperref[prob:es-verdadero]{\textsc{EsVerdadero}} para todos los sistemas lógicos derivados del sistema formal \hyperref[sf:suma-binaria]{\textsc{SumaBinaria}} (incluyendo la decidibilidad del problema \hyperref[prob:es-teorema]{\textsc{EsTeorema}} de este sistema formal) \vspace{4pt} & sí \\

\texttt{turing.py} & Este programa incluye una clase \texttt{Turing} que encapsula la simulación de \texttt{simula\_turing.py} \vspace{4pt} & sí \\

\texttt{universal.py} & Este programa hace el problema \hyperref[prob:universal]{\textsc{Universal}} semidecidible. Esto no es probado en el trabajo, pero se deja aquí para permitir la experimentación \vspace{4pt} & sí \\

\texttt{universal\_a\_parada.py}$^\lozenge$ \linebreak \small{(\cref{lst:universal-a-parada})} & Implementa la reducción de \hyperref[prob:universal]{\textsc{Universal}} a \hyperref[prob:parada]{\textsc{Parada}} \vspace{4pt} & no [NE10] \\

\texttt{utilidades.py} & Este programa implementa funciones que se usan en el resto de programas (``utilidades'') \vspace{2pt} & sí \\

\midrule
\end{tabularx}
\vspace*{-0.2cm}
\begin{tabla}
\caption*{Tabla A.1. (cont.): Índice de programas usados en este trabajo}
\label{tab:indice-programas}
\end{tabla}

Los programas indicados con $^\lozenge$ están basados en programas de \cite{MacCormick2018}; en concreto:

\begin{itemize}
    \item Los programas \texttt{c\_diagonal.py} y \texttt{diagonal.py} son inmediatos de implementar, y son prácticamente idénticos a los programas de la figura 3.7 de \cite{MacCormick2018}.
    \item El programa \texttt{es\_teorema\_peano.py} se basa en el programa de la figura 16.5 de \cite{MacCormick2018}.
    \item El programa \texttt{godel\_peano.py} se basa en el programa de la figura 16.8 de \cite{MacCormick2018}. Usamos la nomenclatura en \emph{strings} de la artimética de Peano que definimos en la \cref{tab:peano-string}.
    \item El programa \texttt{ignora\_entrada.py} se basa en el programa de la figura 6.9 de \cite{MacCormick2018}.
    \item El programa \texttt{maquina\_universal\_parada.py} se basa en el primer programa de la figura 7.10 de \cite{MacCormick2018}. Devolvemos el valor \palabra{sí} para hacer que el programa sea de decisión.
    \item El programa \texttt{parada\_en\_vacio\_a\_es\_verdadero\_peano.py} se basa en el programa de la figura 16.6 de \cite{MacCormick2018}.
    \item El programa \texttt{suma\_binaria.py} usa las expresiones regulares que figuran en el programa \texttt{binAd.py} de la librería de código de \cite{MacCormick2018}. Además, se incluye el código correspondiente a las reglas del resto de sistemas lógicos del capítulo, y se reestructura el código para poder evaluar las fórmulas en cualquiera de estos sistemas.
    \item El programa \texttt{universal\_a\_parada.py} se basa en el segundo programa de la figura 7.10 de \cite{MacCormick2018}. Reestructuramos el código para hacerlo más legible.
\end{itemize}

\subsection*{Justificación de programas no ejecutables}

Algunos de los programas que aparecen en la carpeta \texttt{codigo} no son ejecutables. Los motivos de esto se explican a continuación.

\begin{enumerate}[label={[}NE\arabic*{]},wide = 0pt,widest={100}, leftmargin =*]
    \item Las funciones \texttt{parada\_a\_S} y \texttt{es\_prueba\_S} no están implementadas, pues corresponden a un sistema lógico cualquiera (sería necesario crear una implementación concreta para cada sistema lógico), pero su existencia es probada.
    \item La función \texttt{d} no existe (este programa es ilustrativo). Basta probar a sustituir \texttt{d} por cualquier otro programa de decisión.
    \item La función \texttt{es\_prueba\_peano} no está implementada (véase [NI1]).
    \item La función \texttt{es\_prueba\_S} no está implementada (véase [NE1]).
    \item Las funciones \texttt{es\_teorema\_G} y \texttt{parada\_a\_G} no están implementadas, pues corresponden a un sistema formal cualquiera (sería necesario crear una implementación concreta para cada sistema lógico), pero su existencia se supone.
    \item La función \texttt{parada\_a\_peano} no está implementada (véase [NI2]).
    \item No puede escribirse dado que \textsc{AdivinaConsistente} es no decidible (y no está implementado).
    \item Es una reducción en la que \texttt{parada\_en\_vacio} es un oráculo.
    \item Es una reducción en la que \texttt{es\_verdadero\_peano} es un oráculo, y donde \texttt{parada\_a\_peano} no está implementada (véase [NI2]), pero NO es un oráculo (su existencia se prueba).
    \item Es una reducción en la que \texttt{parada} es un oráculo (de hecho no está implementada porque no es computable).
\end{enumerate}

\subsection*{Justificación de programas no implementados}

Dos de los programas a los que hacemos referencia en este trabajo no han sido implementados. A continuación se concreta el por qué.

\begin{enumerate}[label={[}NI\arabic*{]},wide = 0pt,widest={NI1,NI}, leftmargin =*]
    \item[{[}NI1, NI2{]}] Implementar estas funciones sería un trabajo muy arduo, que se deja como posible en un futuro (véase el \cref{ch:trabajo-futuro}).
\end{enumerate}

\endinput

% Añadir tantos apéndices como sea necesario 

% --------------------------------------------------------------------
% GLOSARIO: Opcional
% --------------------------------------------------------------------

%
\chapter{List of Acronyms}

\begin{multicols}{2}
\begin{description}
    \item[AE] Autoencoder
    \item[AI] Artificial Intelligence
    \item[API] Application Programming Interface
    \item[AST] Abstract Syntax Tree
    \item[BM25] Best Matching 25
    \item[C3] Propane / Propylene mixture
    \item[C4] Butane / Butylene mixture
    \item[CaP] Code-as-Policy
    \item[CoC] Chain-of-Code
    \item[CoT] Chain-of-Thought
    \item[CPS] Cyber-Physical Systems
    \item[CSV] Comma-Separated Values
    \item[DL] Deep Learning
    \item[DSL] Domain Specific Language
    \item[DSR] Design Science Research
    \item[FSM] Finite State Machine
    \item[GAN] Generative Adversarial Network
    \item[HITL] Human-in-the-Loop
    \item[HNSW] Hierarchical Navigable Small World
    \item[IE] Information Extraction
    \item[IoT] Internet of Things
    \item[JSON] JavaScript Object Notation
    \item[LGO] Light Gas Oil
    \item[LLM] Large Language Model
    \item[LPG] Liquefied Petroleum Gas
    \item[LSTM] Long Short-Term Memory
    \item[ML] Machine Learning
    \item[MRL] Matryoshka Representation Learning
    \item[MTEB] Massive Text Embedding Benchmark
    \item[NER] Named Entity Recognition
    \item[NLP] Natural Language Processing
    \item[OCR] Optical Character Recognition
    \item[P\&ID] Piping and Instrumentation Diagram
    \item[PDF] Portable Document Format
    \item[PoT] Program-of-Thoughts
    \item[PSM] Process Safety Management
    \item[PSV] Pressure Safety Valve
    \item[RAG] Retrieval-Augmented Generation
    \item[RNN] Recurrent Neural Network
    \item[RoPE] Rotary Positional Embeddings
    \item[SAG] Search-Augmented Generation
    \item[SQL] Structured Query Language
    \item[TIC] Temperature Indicator Controller
    \item[XML] Extensible Markup Language
\end{description}
\end{multicols} 

% -------------------------------------------------------------------
% BACKMATTER
% -------------------------------------------------------------------
\cleardoublepage
\phantomsection

\backmatter % Desactiva la numeración de los capítulos
\cleardoublepage
\phantomsection

\begin{small} % Normalmente la bibliografía se imprime en un tamaño de letra más pequeño.

\cleardoublepage
\phantomsection

\part*{Referencias e índices}
\addcontentsline{toc}{part}{Referencias e índices}

\phantomsection
\vspace*{1.5cm}
{\sffamily \LARGE \textbf{Bibliografía}}\\[8pt]
\thispagestyle{plain}
\addcontentsline{toc}{chapter}{Bibliografía}
\markboth{Bibliografía}{Bibliografía}

\printbibliography[heading=subbibliography,heading=none]

\end{small}

%\printindex
%\markboth{Índice alfabético}{Índice alfabético}

%\phantomsection 
%\listoffigures
%\markboth{Lista de figuras}{Lista de figuras}

%\phantomsection 
%\listoftablas

%\phantomsection 
%\listofproblemas

%\cleardoublepage
%\phantomsection
%\listoftheorems[ignoreall,show={sistemaformal,sistemalogico},title={Índice de sistemas formales y lógicos}]\clearpage
%\addcontentsline{toc}{chapter}{Índice de sistemas formales y lógicos}

\end{document}
