% =============================================================================
% CHAPTER 4: METHODOLOGY AND DESIGN
% =============================================================================

\chapter{Methodology and Design}\label{methodology-and-design}

\emph{Briefly state the approach when designing and implementing the
main solution of this work. Dual engineering+research-based approach}

\section{Requirements}\label{requirements}

\begin{itemize}
\tightlist
\item
  Functional requirements (input/output, deterministic output, rule
  format, validation)
\item
  Non-functional requirements (accuracy, traceability, modularity,
  scalability, low-latency execution)
\item
  Derived from Repsol use case and general industrial needs
\end{itemize}

\section{Design Principles}\label{design-principles}

\emph{Show design principles followed and why they are important} -
Modularity and composability (why important, protocol-based design) -
Separation of concerns * \emph{LLM Role:} Translation (Text \(\to\)
Code). * \emph{System Role:} Verification (Code \(\to\) Safety). *
\emph{Human Role:} Approval. - Decoupling of Logic and State: extracted
rule (Logic) must be stateless; the execution engine (State) handles the
memory buffers - Traceability and explainability (from documents to
rules)

\section{Research Principles (this is a research problem!)}\label{research-principles-this-is-a-research-problem}

\emph{We want to show that every step of the pipeline is based on a
research approach, evaluating alternatives and checking consistency and
results in order to find the best options for designing the system}

(Ablation Study Design? Think with and without and isolate contributions
to performance, might be hard do to this given the time)

