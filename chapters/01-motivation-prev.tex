


\begin{enumerate}
    \item \textbf{Method Gap:} information extraction tools extract facts (static), not logic (dynamic procedures).
\end{enumerate}

\begin{itemize}
\item
  Industrial process monitoring and anomaly detection challenges,
  knowledge extraction challenge (from an \emph{epistemological}
  perspective, it's not just a ``messy PDF'' problem)

  \begin{itemize}
  \item
    Dark data: industry-oprational knowledge that exists but is
    unqueryable and unexecutable
  \item
    Runtime gap: even if rules are extracted, they are often expressed
    in natural language complexities (e.g., `rolling average') that are
    computationally expensive to compute on historical data. A bridge is
    needed between semantic rule definition and efficient stream
    processing
  \end{itemize}
\item
  \textbf{Case Study Introduction}: C3/C4 splitter separation at Repsol

  \begin{itemize}
  
  \item
    Process description (what can be shared publicly)
  \item
    Why this is a representative problem: complexity, usage
  \item
    The knowledge documentation challenge
  \end{itemize}
\item
  The gap: No consistent automated rule extraction framework from
  technical documents
\end{itemize}

\section{Problem Statement}\label{problem-statement}

\begin{itemize}

\item
  Knowledge accessibility: Knowledge trapped in unstructured
  documentation (PDFs, manuals, specs)
\item
  Domain experts required for rule definition (expensive, not scalable)
\item
  Translation gap: there's no formal grammar mapping natural language
  (ambiguous, context-heavy) to control logic (binary, precise)
\item
  Silent failure risk: manual rule creation is prone to human error that
  goes undetected until a safety incident occurs. Automated extraction
  offers \emph{systematic} verification
\item
  Usefulness and application of these rules (control of processes,
  alarms, tracking of ideal behavior\ldots)
\item
  Need for: automated extraction, explainability, maintainability,
  traceability
\end{itemize}

\section{Research Questions}\label{research-questions}

\emph{These questions will be addressed and referenced throughout the
whole work}

\begin{itemize}

\item
  \emph{RQ1 (Feasibility)}: How can operational knowledge embedded in
  unstructured industrial documentation be automatically extracted and
  formalized into executable rules? / To what extent can Large Language
  Models serve as \textbf{semantic compilers} for translating
  unstructured industrial specifications into executable logic?

  \begin{itemize}
  
  \item
    Opens up: is automated extraction feasible? what representations
    work? what extraction methods? what are the challenges?
  \item
    Answer: proposed method (LLMs + multi-stage pipeline with retrieval
    and grounding + Python functions)
  \end{itemize}
\item
  \emph{RQ2 (Ambiguity)}: What are the fundamental challenges in
  translating natural language process specifications into
  machine-executable monitoring rules, and how can they be addressed? /
  What are the key technical barriers to automating industrial rule
  extraction, and what approaches can address them? / How can the
  \textbf{ontological gap} between natural language descriptions and
  precise sensor instrumentation be bridged without extensive manual
  labeling?

  \begin{itemize}
  
  \item
    Opens up: sensor ambiguity and retrieval, time expressions,
    validation, domain knowledge gaps
  \item
    Answer: proposed method (sensor resolution, time parsing, grounding,
    verification)
  \end{itemize}
\item
  \emph{RQ3 (Reasoning)}: To what extent can large language models
  reason about industrial processes? / What role can external knowledge
  and context play in improving the accuracy and completeness of
  LLM-based rule extraction?

  \begin{itemize}
  
  \item
    Opens up: do LLMs understand industrial concepts? do they need
    context or grounding?
  \item
    Answer: RAG + grounding approach proposed, interesting to talk about
    RAG vs.~fine-tuning here and general-purpose (foundational) models
  \end{itemize}
\item
  \emph{RQ4 (Explainability and trust)}: What properties must an
  automated rule extraction system possess to be trustworthy and
  deployable in industrial settings? / How can automatically extracted
  rules achieve the level of trustworthiness required for industrial
  deployment?

  \begin{itemize}
  
  \item
    Opens up: explainability, traceability; as future work: HITL and
    human validation/oversight
  \item
    Answer: traceability, explainability in extraction and consolidation
  \end{itemize}
\item
  \emph{RQ5 (Quality)}: How can the quality and consistency of
  automatically extracted rules be assessed and improved without
  extensive manual review?

  \begin{itemize}
  
  \item
    Opens up: quality metrics, redundancy detection, consolidation
  \item
    Answer: consolidation workflow
  \end{itemize}
\item
  \emph{RQ6 (Complexity)}: How can the computationally expensive logic
  inherent in natural language specifications (e.g., sliding windows,
  statistical aggregations) be executed efficiently in real-time
  industrial environments?

  \begin{itemize}
  
  \item
    Opens up: streaming framework and rule visualization
  \end{itemize}
\end{itemize}

TODO: To what extent can this approach generalize to real-world
industrial use cases? (not as research question, but in conclusions)

\section{Research Approach and Objectives}\label{research-approach-and-objectives}

\textbf{Research approach:}

\begin{itemize}
    \item Iterative design methodology:
\textbf{``Design Science Research'' (DSR)}
    \item Focus on modular, swappable
components: analyze implementation options and justify chosen one, for
each component
    \item Validation with real industrial documentation
\end{itemize}

\textbf{Primary Objective:} Design and validate an end-to-end framework
for automated extraction of operational monitoring rules from
unstructured industrial documentation

\textbf{Specific Objectives:}

\begin{itemize}
    \item \emph{O1}: Develop a rule extraction
methodology that bridges the gap between natural language specifications
and executable code
    \item \emph{O2}: Investigate and implement mechanisms to
overcome LLM limitations in domain-specific industrial contexts
    \item \emph{O3}: Address the semantic ambiguity inherent in industrial
documentation
    \item \emph{O4}: Establish quality assurance mechanisms for
automatically extracted rules
    \item \emph{O5}: Ensure trustworthiness
through explainability and traceability
    \item \emph{O6}: Efficiently execute
rules in an industrial context
    \item \emph{O7}: Validate the approach on
mock and on real-world industrial use cases
    \item \emph{O8}: Design a scalable, production-ready system architecture
    \item \emph{O9}: Establish
evaluation methodology and conduct systematic assessment
\end{itemize}


