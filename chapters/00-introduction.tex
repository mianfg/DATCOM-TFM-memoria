% =============================================================================
% CHAPTER 0: INTRODUCTION
% =============================================================================

\chapter{Introduction}\label{introduction}

\begin{itemize}
    \item \textbf{Context:} Industry 4.0 requires constant monitoring for safety and efficiency
    \item \textbf{Limits of Data-Driven Detection (Anomaly Detection):} standard anomaly detection usually relies on historical data (ML), but in critical industries that data might not exist because we stop machines because they break, also they are ``black box'' models that lack explainability/trust.
    \item \textbf{Shift to Definition}: previous point orients us towards relying on rule based systems (definition) rather than pure statistical / ML detection.
    \item \textbf{Definition Bottleneck:} we can define anomalies using expert rules, but this process is manual and unascalable.
    \item \textbf{Dark Data:} knowledge exists in manuals, but it's inaccessible to automated systems
    \item \textbf{The Core Thesis:} we propose automated definition via ``Semantic Compilation'' -- translating unstructured technical documents into executable monitoring rules.
    \item \textbf{Roadmap of Thesis:} brief roadmap of the chapters to follow.
\end{itemize}