
\section{Methodological Framework: Design Science Research (DSR)}

This thesis adopts the \textbf{Design Science Research (DSR)} methodology \cite{hevner_design_2004}, a research paradigm suited for engineering contexts where the objective is to create a novel artifact to address a specific domain problem.

The research is structured around the three core DSR cycles:
\begin{enumerate}
    \item \textbf{Relevance Cycle:} Defining the research requirements based on the specific constraints of the industrial environment (e.g., safety, determinism, lack of data).
    \item \textbf{Design Cycle:} An iterative process of building, evaluating, and refining the artifact.
    \item \textbf{Rigor Cycle:} Grounding the design decisions in the state of the art to ensure the solution advances the knowledge base.
\end{enumerate}




\section{Requirements}\label{requirements}

\begin{itemize}

\item
  Functional requirements (input/output, deterministic output, rule
  format, validation)
\item
  Non-functional requirements (accuracy, traceability, modularity,
  scalability, low-latency execution)
\item
  Derived from Repsol use case and general industrial needs
\end{itemize}

\section{Architecture Principles}\label{design-principles}

\emph{Show design principles followed and why they are important}

\begin{itemize}
    \item Modularity and composability (why important, protocol-based design)
    \item Separation of concerns
    \begin{enumerate}
        \item \emph{LLM Role:} Translation (Text \(\to\)
Code).
        \item \emph{System Role:} Verification (Code \(\to\) Safety).
        \item \emph{Human Role:} Approval.
    \end{enumerate}
    \item Decoupling of Logic and State: extracted
rule (Logic) must be stateless; the execution engine (State) handles the
memory buffers
    \item Traceability and explainability (from documents to
rules)
\end{itemize}

\section{Research Principles (this is a research problem!)}\label{research-principles-this-is-a-research-problem}

\emph{We want to show that every step of the pipeline is based on a
research approach, evaluating alternatives and checking consistency and
results in order to find the best options for designing the system}

(Ablation Study Design? Think with and without and isolate contributions
to performance, might be hard do to this given the time)

% =============================================================================
% CHAPTER 2: METHODOLOGY AND DESIGN
% =============================================================================

\chapter{Methodology}\label{methodology-and-design}

\emph{Briefly state the approach when designing and implementing the
main solution of this work. Dual engineering+research-based approach}

\section{Phase 1: Problem Identification \& State of the Art Analysis}

\begin{itemize}
    \item \textbf{Pre-Implementation Analysis:} Before writing any code, the current landscape was analyzed.
    \item \textbf{The Decision for LLMs:} Explain that based on the literature review, Large Language Models were selected as the core technology because previous methods (Regex, ML, Standard NLP) failed at logical reasoning. This was a research-backed decision, not an assumption.
    \item \textbf{Gap Identification:} This phase concluded that while LLMs are promising, they lack the specific architecture needed for safety-critical industry and to satisfy objectives.
\end{itemize}

\section{Phase 2: Iterative Design of Rule Extraction Framework}

Instead of grouping things into arbitrary ``cyycles,'' we describe the ``micro-cycle'' applied to every single component of the system.

The following iterative loop was applied:

\begin{enumerate}
    \item \textbf{Hypothesis/Problem Formulation:} Identify the specific barrier (e.g., ``The model hallucinates sensor tags'').
    \item \textbf{Ablation Study of Options:} Review literature and implementation possibilities (e.g., Option A: Vector Search vs. Option B: Generative Linking).
    \item \textbf{Implementation \& Justification:} Select the best option for our problem, based on theoretical and practical reasons, and with our objectives in mind.
    \item \textbf{Evaluation \& Possible Improvements:} Test the component, observe the results (e.g., ``It works for text but fails for tables''), and define the next problem.
\end{enumerate}

\section{Phase 3: Implementation of Evaluation Framework}



You are completely correct. I am crossing the line between *Methodology* (the approach) and *Implementation* (the solution). The Methodology chapter should define the **rules of the game**, not the final score.

It should describe **how you make decisions**, not **what decisions you made**.

Here is the revised **Chapter 3**, rewritten to be abstract and procedural. It defines the *strategies* you will use to solve the problems identified in Chapter 2, without revealing the specific libraries or algorithms (River, O(1), Twin Document) you ended up building.