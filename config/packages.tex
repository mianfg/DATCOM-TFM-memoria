% !TeX root = ../main.tex
% !TeX encoding = utf8
%
% PACKAGE LOADING
% This file loads all packages used in the document.
% Organized by category for clarity.

\usepackage{svg}

% =============================================================================
% ENCODING AND LANGUAGE
% =============================================================================
\usepackage[utf8]{inputenc}

% Language configuration: English and Spanish (Spanish is default)
% Switch languages with: \selectlanguage{english} or \selectlanguage{spanish}
\usepackage[english]{babel}
\usepackage{csquotes}

% Options:
%   es-nodecimaldot: Keep decimal point (not comma) in math mode
%   es-noindentfirst: Don't indent paragraphs after titles
%   es-tabla: Use "Tabla" instead of "Cuadro" for table environment

% Remove accents from math operators (\lim, \max, etc.)

% =============================================================================
% MATHEMATICS
% =============================================================================
\usepackage{amsmath, amsthm, amssymb, etoolbox, thmtools}
\usepackage{mathtools}
\mathtoolsset{showonlyrefs=true}  % Only number equations that are referenced
\usepackage[mathscr]{eucal}       % \mathscr for script fonts (keeps \mathcal)

% =============================================================================
% TYPOGRAPHY
% =============================================================================
\usepackage[activate={true,nocompatibility},final,tracking=true,kerning=true,spacing=true,factor=1100,stretch=10,shrink=10]{microtype}

% Font selection (UGR style guide):
%   Normal font:      URW Palladio
%   Sans-serif font:  Gill Sans (Cabin)
%   Monospace font:   Inconsolata (loaded via zi4 below)
\usepackage[LGR, T1]{fontenc}
\usepackage[sc, osf]{mathpazo}
\linespread{1.05}
\usepackage[scaled=.95,type1]{cabin}
% Alternative if cabin fails: \renewcommand{\sfdefault}{iwona}
% Note: inconsolata loaded via zi4 package for code listings

% List formatting
\usepackage{enumitem}
\setitemize{itemsep=4pt,topsep=4pt,parsep=4pt,partopsep=0pt}

% Other typography packages
\usepackage{setspace}
\usepackage{changepage}   % for adjustwidth environment

% =============================================================================
% TABLES AND FIGURES
% =============================================================================
\usepackage{booktabs}
\usepackage{tabularx}
\newcolumntype{L}[1]{>{\hsize=#1\hsize\RaggedRight} X}

\usepackage{caption}
\usepackage{float}
\usepackage{rotating}
\usepackage{stfloats,framed}

% =============================================================================
% GRAPHICS AND DIAGRAMS
% =============================================================================
\usepackage[table]{xcolor}
\usepackage{graphicx}
\usepackage{tikz}
\usetikzlibrary{shapes.geometric, arrows.meta, positioning, calc}
\usepackage{tikz-cd}
\usepackage{svg}
\usepackage{pdflscape}
\usepackage{mathabx}

% Set default image directory
\graphicspath{{img/}}

% =============================================================================
% CODE LISTINGS
% =============================================================================
\usepackage{listings}
\usepackage{lstautogobble}
\usepackage{zi4}

% =============================================================================
% CUSTOM FLOATS
% =============================================================================
\usepackage{newfloat}
\DeclareFloatingEnvironment[listname={Índice de problemas},placement=H]{problema}
\DeclareFloatingEnvironment[listname={Index of Tables},placement=H]{tabla}
\DeclareFloatingEnvironment[listname={Index of prompts},placement=H]{prompt}
\DeclareFloatingEnvironment[listname={Index of threads},placement=H]{thread}
\DeclareFloatingEnvironment[listname={List of Programs},placement=H]{program}
\DeclareFloatingEnvironment[listname={Index of rules},name=Rule,placement=H]{generatedrule}

\DeclareCaptionFormat{cont}{#1 (cont.)#2#3\par}

% =============================================================================
% UTILITIES
% =============================================================================
\usepackage{dirtree}
\usepackage{xspace}  % Smart spacing for custom commands

\usepackage[most]{tcolorbox}

% ============================================
% LLM INTERACTION STYLE SETUP
% ============================================

\usepackage[most]{tcolorbox}
\usepackage{fontawesome5}

% -------- Global LLM box style ---------
\tcbset{
  llmstyle/.style={
    enhanced,
    boxrule=0pt,
    left=10pt,
    right=10pt,
    top=8pt,
    bottom=8pt,
    sharp corners,
    borderline west={3pt}{0pt}{#1!80},
    colback=#1!5,
    colframe=#1!60,
    fonttitle=\bfseries,
    before skip=10pt,
    after skip=10pt,
  }
}

% -------- Reusable Commands ---------

\newcommand{\systemmsg}[1]{%
  \begin{tcolorbox}[llmstyle=gray,title=\faCog\ System Message]
  #1
  \end{tcolorbox}
}

\newcommand{\usermsg}[1]{%
  \begin{tcolorbox}[llmstyle=blue,title=\faUser\ Prompt]
  #1
  \end{tcolorbox}
}

\newtcolorbox{UserMessage}[1][]{%
  enhanced,
  colback=purple!5,
  colframe=purple!60,
  borderline west={3pt}{0pt}{purple!80},
  fonttitle=\bfseries,
  title=\faUser\hspace{0.6em}\textsf{User Message (Prompt)},
  #1
}


\newtcolorbox{ToolMessage}[1][]{%
  enhanced,
  colback=cyan!5,
  colframe=cyan!60,
  borderline west={3pt}{0pt}{cyan!80},
  fonttitle=\bfseries,
  title=\faTools\hspace{0.6em}\textsf{Tool Call},
  #1
}

\newcommand{\toolcall}[1]{%
  \begin{tcolorbox}[llmstyle=orange,title=\faTools\ Tool Call]
  \texttt{#1}
  \end{tcolorbox}
}

\newcommand{\toolresponse}[1]{%
  \begin{tcolorbox}[llmstyle=yellow,title=\faRobot\ Tool Response]
  \texttt{#1}
  \end{tcolorbox}
}

\newcommand{\assistantmsg}[1]{%
  \begin{tcolorbox}[llmstyle=green,title=\faCommentDots\ AI Message]
  #1
  \end{tcolorbox}
}

\newtcolorbox{AssistantMessage}[1][]{%
  enhanced,
  colback=teal!5,
  colframe=teal!60,
  borderline west={3pt}{0pt}{teal!80},
  fonttitle=\bfseries,
  title=\faRobot\hspace{0.6em}\textsf{AI Message (Response)},
  #1
}



\newtcolorbox{RuleMessage}[1][]{%
  enhanced,
  colback=orange!5,
  colframe=orange!60,
  borderline west={3pt}{0pt}{orange!80},
  fonttitle=\bfseries,
  title=\faCog\hspace{0.6em}\textsf{Rule},
  #1
}

\newtcolorbox{CodeListing}[1][]{%
  enhanced,
  colback=lightgray!5,
  colframe=lightgray!60,
  borderline west={3pt}{0pt}{lightgray!80},
  #1
}



% --- simple badge (always same color)
\newcommand{\rulebadge}[1]{%
  \tcbox[
    on line,
    colback=black!10,
    colframe=black!40,
    arc=3pt,
    boxrule=0.6pt,
    left=6pt,right=6pt,top=2pt,bottom=2pt,
    fontupper=\sffamily\bfseries\footnotesize
  ]{\textsf{#1}}%
}

% --- Python code style
\lstdefinestyle{RulePythonStyle}{
    language=Python,
    basicstyle=\ttfamily\small,
    breaklines=true,
    frame=none,
    autogobble=true,
    columns=fullflexible
}

% --- Visual rule card
% Usage: \begin{rulecard}{rule_name}{rule_type} ... \end{rulecard}
\newtcolorbox{rulecard}[2][]{%
  enhanced,
  before skip=12pt, after skip=12pt,
  colback=white,
  colframe=black!8,
  boxrule=0.8pt,
  arc=4mm,
  drop shadow=black!30!white,
  width=\linewidth,
  leftrule=0pt, rightrule=0pt, toprule=0pt, bottomrule=0pt,
  boxsep=6pt,
  fonttitle=\sffamily\Large\bfseries,
  title={\faCodeBranch\quad \texttt{#1}},        % rule name (monospaced)
  after title={\hfill \rulebadge{#2}},           % right-aligned badge
  colbacktitle=black!5,
  coltitle=black,
  attach boxed title to top left={yshift=-3mm,xshift=3mm},
  boxed title style={boxrule=0pt,arc=3mm,boxsep=2pt},
  #1
}



\usepackage{draftwatermark}
\SetWatermarkText{\sffamily DRAFT 3}
\SetWatermarkColor[gray]{0.9}  % 0 = black, 1 = white → 0.5 = 50% gray
\SetWatermarkAngle{60}
\SetWatermarkScale{1.6}




% =============================================================================
% INDEX
% =============================================================================
\usepackage{makeidx}
\makeindex

% =============================================================================
% BIBLIOGRAPHY
% =============================================================================
\usepackage[backend=biber,style=apa,minbibnames=1,maxbibnames=1,maxcitenames=2,citestyle=numeric]{biblatex}
\addbibresource{references.bib}

% =============================================================================
% CROSS-REFERENCE UTILITIES
% =============================================================================
\usepackage{xurl}
% Note: cleveref must be loaded AFTER hyperref (see config/hyperref.tex)

