% !TeX root = ../libro.tex
% !TeX encoding = utf8

\setchapterpreamble[c][0.75\linewidth]{%
	\sffamily
  \emph{Sabemos que cada época tiene sus propios problemas, que la siguiente resuelve o desecha por inútiles y sustituye por otros nuevos.}
 \begin{flushright} — David Hilbert, \cite{Hilbert1902} \end{flushright}
% \\[8pt]
	\par\bigskip
}
\vspace{28pt}

\chapter{Trabajo futuro}\label{ch:trabajo-futuro}

A pesar de que hemos logrado cumplir los objetivos estipulados para este trabajo, es posible continuar la investigación computacional de los teoremas de incompletitud, así como mejorar e implementar algunos aspectos.

\subsubsection*{Segundo Teorema de Incompletitud}

Sería interesante realizar un estudio computacional del Segundo Teorema de Incompletitud. Para ello, sería necesario encontrar una demostración de dicho teorema a través de programas. En el proceso de investigación y desarrollo de este trabajo, se logró probar un resultado sintáctico usando las mismas técnicas que la demostración de Gödel, por lo que se decidió no incluirlo. En un futuro, sería de interés probar resultados sintácticos y semánticos para el Segundo Teorema de Incompletitud a través de programas.

\subsubsection*{Mejorar demostración de la versión sintáctica}

La demostración del \cref{teo:incompletitud-peano} es constructiva, es decir, no sólo garantizamos la existencia de una fórmula verdadera pero no demostrable, sino que encontramos una en concreto (\texttt{no\_para\_peano}). Más adelante generalizamos el resultado en el \cref{teo:incompletitud-semantica}. Se podría encontrar una demostración constructiva similar para el caso semántico, ya que la demostración del \cref{teo:incompletitud-sintactica}, a partir de \textsc{AdivinaConsistente}, no es constructiva.

\subsubsection*{Implementar \texttt{es\_teorema\_peano.py} y \texttt{parada\_a\_peano.py}}

En el \cref{ch:sistemas-logicos} introducimos la aritmética de Peano (\cref{sl:peano}), y argumentamos la existencia de los programas \texttt{es\_prueba\_peano.py} y \texttt{parada\_a\_peano.py}. La implementación de estos programas se omite debido a su laboriosidad, pero sería posible hacerlo en un futuro.

\endinput
